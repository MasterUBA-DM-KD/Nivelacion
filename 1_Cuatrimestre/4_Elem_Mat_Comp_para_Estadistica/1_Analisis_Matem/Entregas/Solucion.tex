\documentclass{article}
\usepackage[utf8]{inputenc}
\usepackage{graphicx}
\usepackage{amsmath}
\usepackage{float}


\title{
Clase 3: Ejercicio grupal - G10

Curso Nivelador Análisis – Carrera de Especialización en Estadística 2022
}

\author{Alejandro Uribe}
\date{Octubre 2022}

\begin{document}

\maketitle

\begin{enumerate}
    \item Dada $f(x) = \frac{1}{\sqrt{x^2-9}}$ con $x \in [-1, 2]$
          \begin{enumerate}
              \item Calcular $f'$, $Domf'$ y determinar los puntos críticos.
              \item Máximos y mínimos absolutos de $f$ en $[-1, 2]$.
          \end{enumerate}
          A fin de calcular $f'$, $Domf'$, puntos críticos de $f(x)$, y máximos y mínimos absolutos de $f(x)$, la función debe ser derivable en el intervalo $x\in[-1,2]$.

          El $Domf$, son aquellos valores de $x$ para los cuales existe un valor asociado de la función $f(x)$:
          $$Domf = \{ x \in \mathbf{R}\ / \quad \exists f(x)\}$$

          Se calcula el $Domf$:
          $$Domf = \{ x \in \mathbf{R}\ / \quad x^2-9 > 0\}$$
          Es decir,
          $$Domf = \{ x \in \mathbf{R}\ / \quad x<-3 \quad | \quad x>3 \}$$

          Por lo tanto, $f(x)$ no se encuentra definida en el intervalo $[-1,2]$ y no es derivable en tal intervalo.
    \item Dada $f(x) = \frac{1}{\sqrt{x^2-9}}$
          \begin{enumerate}
              \item Hallar el $Domf$

                    El $Domf$ son aquellos valores de $x$ para los cuales existe un valor asociado de la función $f(x)$:
                    $$Domf = \{ x \in \mathbf{R}\ / \quad \exists f(x)\}$$

                    Se calcula el $Domf$:
                    $$Domf = \{ x \in \mathbf{R}\ / \quad x^2-9 > 0\}$$

                    Es decir,
                    $$Domf = \{ x \in \mathbf{R}\ / \quad x<-3 \quad | \quad x>3 \}$$

              \item Para cada $c$ en el borde del dominio de $f(x)$, determinar qué límites laterales tiene sentido tomar cuando $x$ tiende a $c$ y calcularlos.

                    El dominio de $f(x)$ es $Domf = (-\infty, -3) \cup (3, \infty)$. Por lo que tiene sentido tomar los límites cuando $x \to \{-3^-, 3^+, \pm \infty\}$.

                    $$\lim_{x\to \pm \infty} \frac{1}{\sqrt{x^2-9}} = 0$$
                    $$\lim_{x\to -3^-} \frac{1}{\sqrt{x^2-9}} = +\infty$$
                    $$\lim_{x\to 3^+} \frac{1}{\sqrt{x^2-9}} = +\infty$$

              \item Calcular $f'$, $Domf'$ y hallar sus puntos críticos. Luego analizar dónde crece y decrece $f$.

                    \textbf{Calcular $f'$}
                    $$\frac{d f(x)}{dx} = \frac{d}{d x} \left(\frac{1}{\sqrt{x^2-9}} \right)$$
                    $$\frac{d f(x)}{dx}= -\frac{x}{(x^2-9)^{\frac{3}{2}}}$$

                    \textbf{Calcular $Domf'$}

                    El $Domf'$, son aquellos valores de $x$ para los cuales existe un valor asociado de la función $f'(x)$:
                    $$Domf' = \{ x \in \mathbf{R}\ / \quad \exists f'(x)\}$$
                    Se calcula el $Domf'$:
                    $$Domf'= \{ x \in \mathbf{R}\ / \quad x^2-9 > 0\}$$
                    Es decir,
                    $$Domf'= \{ x \in \mathbf{R}\ / \quad x<-3 \quad | \quad x>3 \}$$

                    \textbf{Puntos críticos}

                    Los puntos críticos de $f(x)$ son los $x_i$ tales que $f'(x_i) = 0$.
                    $$f'(x) = -\frac{x}{(x^2-9)^{\frac{3}{2}}}$$
                    $$-\frac{x}{(x^2-9)^{\frac{3}{2}}} = 0$$
                    $$\left(\frac{1}{(x^2-9)^{\frac{3}{2}}} \right) (-x) = 0$$
                    Notar que $\left(\frac{1}{(x^2-9)^{\frac{3}{2}}} \right) > 0 \quad \forall x \in Domf'$. Por otro lado, $(-x)$ no toma el valor de cero ya que tal no pertence al $Domf'$. Por lo tanto, $f(x)$ no tiene puntos críticos en su dominio.

                    \textbf{Intervalos $f(x)$ donde crece o decrece}
                    \begin{itemize}
                        \item $f(x)$ crece si $f'(x) > 0$, es decir:
                              $$\left(\frac{1}{(x^2-9)^{\frac{3}{2}}} \right) (-x) > 0$$
                              Notar que $\left(\frac{1}{(x^2-9)^{\frac{3}{2}}} \right) > 0 \quad \forall x \in Domf'$. Por otro lado, $(-x) > 0$ si $x < 0$.
                        \item $f(x)$ decrece si $f'(x) < 0$, es decir:
                              $$\left(\frac{1}{(x^2-9)^{\frac{3}{2}}} \right) (-x) < 0$$
                              Notar que $\left(\frac{1}{(x^2-9)^{\frac{3}{2}}} \right) > 0 \quad \forall x \in Domf'$. Por otro lado, $(-x) > 0$ si $x > 0$.
                    \end{itemize}
                    Por lo tanto,
                    \begin{equation*}
                        \begin{cases}
                            f'(x)> 0, & \text{si $x < -3 \Rightarrow f(x) \uparrow$}  \\
                            f'(x)< 0, & \text{si $x > 3 \Rightarrow f(x) \downarrow$}
                        \end{cases}
                    \end{equation*}

              \item Hallar máximos y mínimos locales y decidir si son absolutos.

                    Del item anterior se concluyó que $f(x)$ no tiene puntos críticos. No obstante, $f(x)$ tiene dos asíntotas verticales: $x=-3, \quad x=3$ y una asíntota vertical en $y=0$.

              \item Esbozar un gráfico de $f$.

                    De los items anteriores se tiene que:
                    \begin{equation*}
                        \text{Asíntotas}\begin{cases}
                            \text{Horizontales}: & \left\{ \text{$y=0$} \right\}             \\
                            \text{Verticales}:   & \left\{ \text{$x=-3, \quad x=3$} \right\}
                        \end{cases}
                    \end{equation*}

                    \begin{equation*}
                        \text{Crecimiento/Decrecimiento}\begin{cases}
                            \text{$f(x) \uparrow$},   & \text{si $x<-3$} \\
                            \text{$f(x) \downarrow$}, & \text{si $x>3$}
                        \end{cases}
                    \end{equation*}

                    Resta analizar la concavidad de $f(x) \quad \forall x \in Domf$. Para ello se calcula la segunda derivada de $f(x)$.

                    $$\frac{d^2f(x)}{dx^2} = \frac{d^2}{dx^2} \left(\frac{1}{\sqrt{x^2-9}} \right)$$
                    $$\frac{d^2f(x)}{dx^2} = \frac{2x^2 - 9}{(x^2 -9)^{\frac{5}{2}}}$$

                    La función $f(x)$ es cóncava hacia abajo si $f"(x) < 0$ y cóncava hacia arriba si $f"(x) > 0$. Al tomar valores arbitarios de $x \in Domf$, se nota que:

                    \begin{equation*}
                        \begin{cases}
                            \text{$f"(x)>0$},   & \text{si \{$x<-3, \quad x>3\}$} \\
                            \text{$f"(x) < 0$}, & \text{en ningún caso}
                        \end{cases}
                    \end{equation*}

                    Se esboza la gráfica de $f(x)$ a continuación.

                    \begin{figure}[H]
                        \centering
                        \includegraphics[width=9cm]{ChartIO}
                    \end{figure}

              \item Hacer un gráfico de la función usando la computadora y comparar con el gráfico del item anterior.
                    \begin{figure}[H]
                        \centering
                        \includegraphics[width=9cm]{image.png}
                    \end{figure}

                    Notar que la función generada por computadora se acerca mas rápido a sus asíntotas verticales y horizontales que en el esbozo a mano.

              \item Decidir si son verdaderas o falsas cada una de las siguientes afirmaciones
                    \begin{enumerate}
                        \item $f(x) \leq 1/3 \quad \forall x \in Domf$
                              Se chequea tal condición:
                              $$\frac{1}{\sqrt{x^2-9}} \leq \frac{1}{3}$$
                              La condición se cumple si:
                              $$x \geq 3\sqrt{2}$$
                              Y se sabe que:
                              $$Domf = \{ x \in \mathbf{R}\ / \quad x<-3 \quad | \quad x>3 \}$$
                              Por lo tanto, la afirmación es falsa.
                        \item $f(x) < 1/2 \quad \forall x \in Domf$
                              Se chequea tal condición:
                              $$\frac{1}{\sqrt{x^2-9}} \leq \frac{1}{2}$$
                              La condición se cumple si:
                              $$x \geq \sqrt{13}$$
                              Y se sabe que:
                              $$Domf = \{ x \in \mathbf{R}\ / \quad x<-3 \quad | \quad x>3 \}$$
                              Por lo tanto, la afirmación es falsa.
                    \end{enumerate}
          \end{enumerate}
\end{enumerate}

\end{document}
