\documentclass[a4paper,spanish]{article}

%% Use utf-8 encoding for foreign characters
\usepackage[T1]{fontenc}
\usepackage[utf8]{inputenc}
\usepackage{babel}

%% Vector based fonts instead of bitmaps
\usepackage{lmodern}

%% Useful
\usepackage{fullpage} % Smaller margins
\usepackage{enumerate}

%% Theorem
\usepackage{amsthm}

%% More math
\usepackage{amsmath}
\usepackage{amssymb}
\usepackage{delarray}
\pagenumbering{arabic}

%% Document Header
\title{Entrega Clase 1 - Álgebra lineal}
\author{Alejandro Uribe}
\date{Octubre 2022}

\begin{document}
\maketitle
\section*{Enunciado}
Determinar los valores de $k$ (si existen) que hacen que el sistema resulte compatible determinado, compatible indeterminado o incompatible.

\[
    \begin{cases}
        x + y + z = 1       \\
        (k+2)x + ky -z  = 0 \\
        -x +y -2z = -1
    \end{cases}
\]

\section*{Desarrollo}

A partir del sistema se construyó la matriz ampliada:

\[
    \begin{array}({@{}ccc|c@{}})
        1   & 1 & 1  & 1  \\
        k+2 & k & -1 & 0  \\
        -1  & 1 & -2 & -1
    \end{array}\
\]

Y se llevó a cabo eliminación gaussiana:

\[
    \begin{array}({@{}ccc|c@{}})
        1   & 1 & 1  & 1  \\
        k+2 & k & -1 & 0  \\
        -1  & 1 & -2 & -1
    \end{array}\
    \;
    \xrightarrow{f_2 - 2*f_1 \rightarrow f_2}
    \;
    \begin{array}({@{}ccc|c@{}})
        1  & 1   & 1  & 1  \\
        k  & k-2 & -3 & -2 \\
        -1 & 1   & -2 & -1
    \end{array}\
    \;
    \xrightarrow{f_3 + f_1 \rightarrow f_3}
    \;
\]
\[
    \begin{array}({@{}ccc|c@{}})
        1 & 1   & 1  & 1  \\
        k & k-2 & -3 & -2 \\
        0 & 2   & -1 & 0
    \end{array}\
    \xrightarrow{f_1 \leftrightarrow f_2}
    \;
    \begin{array}({@{}ccc|c@{}})
        k & k-2 & -3 & -2 \\
        1 & 1   & 1  & 1  \\
        0 & 2   & -1 & 0
    \end{array}\
    \;
    \xrightarrow{f_1 + f_3 \rightarrow f_1}
    \;
\]

\[
    \begin{array}({@{}ccc|c@{}})
        k & k & -4 & -2 \\
        1 & 1 & 1  & 1  \\
        0 & 2 & -1 & 0
    \end{array}\
    \;
    \xrightarrow{k*f_2 \rightarrow f_2}
    \;
    \begin{array}({@{}ccc|c@{}})
        k & k & -4 & -2 \\
        k & k & k  & k  \\
        0 & 2 & -1 & 0
    \end{array}\
    \;
    \xrightarrow{f_2 - f_1 \rightarrow f_2}
    \;
\]

\[
    \begin{array}({@{}ccc|c@{}})
        k & k & -4  & -2  \\
        0 & 0 & k+4 & k+2 \\
        0 & 2 & -1  & 0
    \end{array}\
    \;
    \xrightarrow{f_2 \leftrightarrow f_3}
    \;
    \begin{array}({@{}ccc|c@{}})
        k & k & -4  & -2  \\
        0 & 2 & -1  & 0   \\
        0 & 0 & k+4 & k+2
    \end{array}\
\]

La anterior es una matriz escalonada a la que le corresponde el sistema:
\[
    \begin{cases}
        kx + ky - 4z = -2 \\
        2y - z = 0        \\
        z = \frac{k+2}{k+4}
    \end{cases}
\]

Se remplazó la variable $z$ y se simplificó el sistema:

\[
    \begin{cases}
        kx + ky - 4z = -2 \\
        2y - z = 0        \\
        z = \frac{k+2}{k+4}
    \end{cases}
    \;
    \rightarrow
    \;
    \begin{cases}
        kx = -2 - ky + 4z \\
        y = \frac{z}{2}   \\
        z = \frac{k+2}{k+4}
    \end{cases}
    \;
    \rightarrow
    \;
\]
\[
    \begin{cases}
        kx  = -2 - k \frac{k+2}{2(k+4)} + 4 \frac{k+2}{k+4} \\
        y = \frac{k+2}{2(k+4)}                              \\
        z = \frac{k+2}{k+4}
    \end{cases}
    \;
    \rightarrow
    \;
    \begin{cases}
        x  = \frac{2-k}{2(k+4)} \\
        y = \frac{k+2}{2(k+4)}  \\
        z = \frac{k+2}{k+4}
    \end{cases}
\]

\subsection*{Sistema incompatible}

En este caso el sistema no tiene solución. Notar que el denominador común en todas las variables es $k + 4 = 0$. Dicho denominador debe ser tal que $k+4 \neq 0$, ya que en otro caso existiría una indeterminación, es decir, $k = -4$. Por lo tanto, el sistema en cuestión no tiene solución si $k = -4$.

\subsection*{Sistema compatible determinado}

Este caso implica que el sistema tiene solución única, es decir, si después de escalonar la matriz ampliada de un sistema de $n$ incógnitas se obtuvieron exactamente $n$ ecuaciones no nulas. Ahora bien, el sistema en cuestión cumple dicha condición si se cumple la igualdad:

\[
    \begin{cases}
        x + y + z = 1       \\
        (k+2)x + ky -z  = 0 \\
        -x +y -2z = -1
    \end{cases}
    \text{donde \quad}
    \begin{cases}
        x  = \frac{2-k}{2(k+4)} \\
        y = \frac{k+2}{2(k+4)}  \\
        z = \frac{k+2}{k+4}
    \end{cases}
\]

Del ejercicio anterior se concluyó que existe una indeterminación si $k = -4$. Por lo tanto, el sistema en cuestión tiene solución única si $k\neq-4$. Sea $S$ el conjunto de soluciones tal que:

\[
    S = \left\{
    (x, y, z, k)
    \in
    \left\{ (\frac{2-k}{2(k+4)}, \frac{k+2}{2(k+4)}, \frac{k+2}{k+4}, k) \right\}:
    k \in \mathbb{R} - \{-4\}
    \right\}
\]

\subsection*{Sistema compatible indeterminado}

Este caso implica que el sistema tiene infinitas soluciones. Es decir, si al escalonar la matriz ampliada se obtuvo menos de $n$ filas no nulas en las primeras $n$ columnas, y el resto de las filas son nulas en todas las columnas.

Para el sistema en cuestión, cumpliría este caso si se encuentra un $k$ que haga que los términos $(k+2)$ y $(k+4)$ sean ambos igual a cero. Es decir:

\[
    z (k + 4) = (k + 2)
\]

\[
    z (0) = (0)
\]

No obstante, no existe dicho $k$. Por lo tanto, se concluye que no existe un valor de $k$ para que el sistema en cuestión sea \emph{compatible indeterminado}. Hecho que se puede comprobar gráficamente, siendo $k+4$ y $k+2$ dos líneas paralelas.

\end{document}