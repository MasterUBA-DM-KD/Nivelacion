\documentclass[a4paper,spanish]{article}

%% Use utf-8 encoding for foreign characters
\usepackage[T1]{fontenc}
\usepackage[utf8]{inputenc}
\usepackage{babel}

%% Vector based fonts instead of bitmaps
\usepackage{lmodern}
\usepackage{physics}
\usepackage{gensymb}

%% Useful
\usepackage{fullpage} % Smaller margins
\usepackage{enumerate}

%% Theorem
\usepackage{amsthm}

%% More math
\usepackage{amsmath}
\usepackage{amssymb}
\usepackage{delarray}
\pagenumbering{arabic}
\decimalpoint

%% Document Header
\title{Entrega Clase 2 - Álgebra lineal}
\author{Alejandro Uribe}
\date{Noviembre 2022}

\begin{document}
\maketitle

\section*{Enunciado}
Dados $u = (2, 2, 1)$ y $v = (2, 1, 3)$.
\begin{enumerate}
    \item Hallar $\lVert u \rVert$ y $\lVert v \rVert$
    \item Hallar el ángulo que forman $u$ y $v$.
    \item Hallar la proyección de $w = (1, 1, 1)$ sobre la recta $\langle u \rangle$
    \item Verificar que el conjunto de vectores $\{u, v\}$ es linealmente independiente y hallar por Gram-Schmidt una base ortonormal de $V = \langle u,v \rangle$
    \item Completar la base hallada en el ítem anterior a una base ortonormal de $\mathbf{B}$ de $\mathbb{R}^{3}$
    \item Escribir al vector $(1,0,0)$ como combinación lineal de los vectores de $\mathbf{B}$ y verificar el resultado obtenido.
\end{enumerate}

\section*{Solución}
\subsection*{Norma de los vectores}
Dado un vector $v  \in \mathbb{R}^{n}$, su normal se define como:
\[
    \lVert v \rVert = \sqrt{\sum_{i=1}^{n}v_{i}^2}
\]
Entonces, para los vectores en cuestión:
\begin{align*}
    \lVert u \rVert = \sqrt{\sum_{i=1}^{n}u_{i}^2} = \sqrt{ 2^2 + 2^2 + 1^2 } = \sqrt{9} = 3 \\
    \lVert v \rVert = \sqrt{\sum_{i=1}^{n}v_{i}^2} = \sqrt{ 2^2 + 1^2 + 3^2 } = \sqrt{14}
\end{align*}
\subsection*{Ángulo entre vectores}
El ángulo $\theta$ entre dos vectores $u$ y $v$ se define como:
\[
    \theta = \arccos \left(\frac{\langle u, v \rangle}{\lVert u \rVert \lVert v \rVert} \right) = \arccos \left(  \frac{\displaystyle\sum_{i=1}^{n}u_{i}v_{i}}{\sqrt{\displaystyle\sum_{i=1}^{n}u_{i}^2} \sqrt{\displaystyle\sum_{i=1}^{n}v_{i}^2}}  \right)
\]
Entonces, para los vectores en cuestión:
\[
    \theta = \arccos \left( \frac{2 \cdot 2 + 2 \cdot 1 + 1 \cdot 3}{3 \cdot \sqrt{14}}\right)  \approx 0.6405 \approx 36.7^{\circ}
\]

\subsection*{Proyección de un vector sobre una recta}
La proyección $p$ de un vector $w$ sobre la recta generada por un vector $u$ se define como:
\[
    p = \frac{\langle u, w \rangle}{\langle u, u \rangle}u
\]
Ya que $\lVert u \rVert = \sqrt{\langle u,u \rangle}$
\[
    p = \frac{\displaystyle\sum_{i=1}^{n}u_{i}w_{i}}{\lVert u \rVert^2}u
\]

Entonces, para el caso de $u$ y $w$:
\[
    p = \frac{2 \cdot 1 + 2 \cdot 1 + 1 \cdot 1}{3^2} \left(\begin{matrix}2 \\2\\1\end{matrix}\right) = \frac{5}{9} \left(\begin{matrix}2 \\2\\1\end{matrix}\right)
\]

\subsection*{Independencia lineal}
El conjunto $\{v_1,v_2, ...,v_n\}$ es linealmente independiente si $a_i = 0, \; \forall i: \; 1\leq i \leq n$ tal que $\displaystyle\sum_{i=1}^n a_i v_i = 0$. Para el caso en cuestión:
\[
    a_1u + a_2v = 0
\]

Es decir,

\[
    a_1 \left(\begin{matrix}2 \\2\\1\end{matrix}\right) + a_2 \left(\begin{matrix}2 \\1\\3\end{matrix}\right) = 0
\]

Se procede a resolver el siguiente sistema:

\begin{align*}
    \begin{cases}
        2a_1 + 2a_2 = 0 \\
        2a_1 + a_2 = 0  \\
        a_1 + 3a_2 = 0
    \end{cases}
    \;
    \begin{cases}
        2a_1 + 2a_2 = 0 \\
        a_2 = -2a_1     \\
        a_1 = -3a_2
    \end{cases}
    \;
    \begin{cases}
        2(-3a_2) + 2a_2 = 0
    \end{cases}
    \;
    \begin{cases}
        -6a_2 + 2a_2 = 0
    \end{cases}
    \;
    \begin{cases}
        a_2 = 0
    \end{cases}
\end{align*}

Sabiendo que $a_2 = 0$, si se remplaza en $a_1=-3a_2$ se concluye que $a_i = 0, \; \forall i: \; 1\leq i \leq n$. Por lo tanto, el sistema $\{u,v\}$ es linealmente independiente.

\subsection*{Base ortogonal}
Una base $\mathbf{B}$ ortogonal para $V=\langle u,v \rangle$ se calcula realizando los siguientes pasos:

\begin{align*}
    v_1 = u \\
    v_2 = v - \left(\frac{v_1 \cdot v}{v_1 \cdot v_1}\right) v_1
\end{align*}

Remplazando los valores de $u$ y $v$:
\begin{align*}
    v_1 = \left(\begin{matrix}2 \\2\\1\end{matrix}\right) \\
    v_2 = \left(\begin{matrix}2 \\1\\3\end{matrix}\right) - \left(\frac{\left(\begin{matrix}2 \\2\\1\end{matrix}\right) \cdot \left(\begin{matrix}2 \\1\\3\end{matrix}\right)}{\left(\begin{matrix}2 \\2\\1\end{matrix}\right) \cdot \left(\begin{matrix}2 \\2\\1\end{matrix}\right)}\right) \left(\begin{matrix}2 \\2\\1\end{matrix}\right)
\end{align*}

Tras simplificar:

\begin{align*}
    v_1 = \left(\begin{matrix}2 \\2\\1\end{matrix}\right) \\
    v_2 = \left(\begin{matrix}0 \\-1\\2\end{matrix}\right)
\end{align*}

\subsection*{Base ortonormal}
Resta dividir los vectores sobre su norma para obtener una base ortonormal.

\begin{align*}
    q_1 = \frac{1}{3}\left(\begin{matrix}2 \\2\\1\end{matrix}\right) \\
    q_2 = \frac{1}{\sqrt{5}} \left(\begin{matrix}0 \\-1\\2\end{matrix}\right)
\end{align*}

\subsection*{Vector como combinacion lineal de los vectores de una base}
Dada una base $\mathbf{B}$ ortonormal de un espacio vectorial $V$, un vector $v \in V$ se puede escribir como combinación lineal de los vectores de $\mathbf{B}$.

\[
    v = a_iq_1 + ... + a_nq_n
\]

Cada uno  de los coeficientes $a_i$ se obtiene por la fórmula:

\[
    a_i = \langle v, q_i \rangle
\]

Para el vector en cuestión:

\begin{align*}
    a_1 = \langle v, q_1 \rangle = \frac{1}{3}\left(\begin{matrix}2 \\2\\1\end{matrix}\right) \cdot \left(\begin{matrix}1\\0\\0\end{matrix}\right) \\
    a_2 = \langle v, q_2 \rangle = \frac{1}{\sqrt{5}} \left(\begin{matrix}0 \\-1\\2\end{matrix}\right) \cdot \left(\begin{matrix}1\\0\\0\end{matrix}\right)
\end{align*}
Tras simplificar:
\begin{align*}
    a_1 =  \frac{2}{3} \\
    a_2 = 0
\end{align*}

Se busca confirmar que $v$ pertenece al subespacio $V$ o bien, que es una combinación lineal de la base ortonormal $\mathbf{B}$:

\[
    v = \frac{2}{3} \cdot \frac{1}{3} \cdot \left(\begin{matrix}2 \\2\\1\end{matrix}\right) + 0 \cdot \frac{1}{\sqrt{5}} \cdot \left(\begin{matrix}0 \\-1\\2\end{matrix}\right)
\]

Tras simplificar:

\[
    v = \frac{2}{9} \cdot \left(\begin{matrix}2 \\2\\1\end{matrix}\right)
\]

Se concluye que $v$ no pertenece al subespacio $V$.


\end{document}